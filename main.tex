\documentclass[a4paper,12pt,twoside]{article}
\usepackage{xcolor}
\definecolor{verde}{rgb}{0,0.5,0}
\definecolor{jpurple}{rgb}{0.5,0,0.35}
\usepackage[utf8x]{inputenc}
\usepackage{graphicx}
\usepackage{float}
\usepackage[brazilian]{babel}
\usepackage{indentfirst}
\usepackage{steinmetz}
\usepackage{multicol}
\usepackage[left=2cm, right=2cm, top=2cm]{geometry}
\setlength\parindent{1cm}
\usepackage{mathrsfs, amsmath}
\usepackage{textcomp}
\usepackage{gensymb}
\usepackage{lipsum}
\usepackage{natbib}
\usepackage{listings}
\usepackage{xcolor}



% Definindo novas cores
\definecolor{verde}{rgb}{0.25,0.5,0.35}
\definecolor{jpurple}{rgb}{0.5,0,0.35}
% Configurando layout para mostrar codigos Python

\lstset{
  escapeinside={!}{!}, 
  language=Python,
  basicstyle=\ttfamily\small,
  keywordstyle=\color{jpurple}\bfseries,
  stringstyle=\color{red},
  commentstyle=\color{verde},
  morecomment=[s][\color{blue}]{/**}{*/},
  extendedchars=true,
  showspaces=false,
  showstringspaces=false,
  numbers=left,
  numberstyle=\tiny,
  breaklines=true,
  backgroundcolor=\color{cyan!10},
  breakautoindent=true,
  captionpos=b,
  xleftmargin=0pt,
  tabsize=4
}



\headheight = 10pt

\setcounter{section}{-1}



\date{}

\begin{document}

% capa
\begin{titlepage} %iniciando a "capa"
\begin{center} %centralizar o texto abaixo
{\large Universidade Federal do Rio de Janeiro}\\[0.2cm] %0,2cm é a distância entre o texto dessa linha e o texto da próxima
{\large Escola Politécnica}\\[0.2cm] % o comando \\ "manda" o texto ir para próxima linha
{\large Departamento de Engenharia Eletrônica e de Computação}\\[0.2cm]
{\large Engenharia Eletrônica e de Computação}\\[0.2cm]
{\large Álgebra Linear 2}\\[5.1cm]
{\bf \huge TRABALHO DE}\\ % o comando \bf deixa o texto entre chaves em negrito. O comando \huge deixa o texto enorme
{\bf \huge ÁLGEBRA LINEAR 2}\\[5.1cm] 
\end{center} %término do comando centralizar
{\large Aluna: Karen dos Anjos Arcoverde}\\[0.7cm] % o comando \large deixa o texto grande
{\large Professor: Marcello Luiz Rodrigues de Campos}\\[5.1cm]
\begin{center}
{\large Rio de Janeiro}\\[0.2cm]
{\large 2021}
\end{center}
\end{titlepage} %término da "capa"


\renewcommand{\contentsname}{Sumário}

\tableofcontents
\clearpage





\section{Introdução}
\subsection{Conteúdo}
    O relatório contém os resultados encontrados para cada questão passada pelo professor e o código final em linguagem Python.
\subsection{Software e linguagem}
    O software usado para programação foi o Spyder e a linguagem foi o Python 3.8.
\subsection{Bibliotecas}
As bibliotecas utilizadas para construir o código em Python foram:
\begin{itemize}
   \item Numpy
   \item Scipy
 \end{itemize}
 \subsection{Base de dados}
 O  conjunto de dados "Iris" selecionado para o trabalho foi:
 \begin{table}[H]
\begin{tabular}{|c|c|c|} \hline
\textbf{ESPÉCIES} & DE & PARA \\\hline
\textbf{Iris-setosa}  & 25 & 39 \\\hline
\textbf{Iris-versicolor} & 75 & 89  \\\hline
\textbf{Iris-virginica} & 125 & 139 \\\hline

 
\end{tabular}
\label{tabela2}
\centering
\caption{Base de dados "Iris" selecionada}
\label {tabela2}
\end{table}



\section{Questão 1}
\subsection{Resultados}

\section{Questão 2}
\subsection{Resultados}
\subsubsection {Iris-Setosa}


\begin{lstlisting}
Iris-Setosa

SEM O TERMO INDEPENDENTE: 
V =  [[-0.80618188 -0.45661854  0.37625828]
 [-0.54157827  0.31342349 -0.78003762]
 [-0.23825146  0.8326255   0.49997102]]

!$\Lambda$! =  [[5.86392634e+02 0.00000000e+00 0.00000000e+00]
 [0.00000000e+00 5.29776824e-01 0.00000000e+00]
 [0.00000000e+00 0.00000000e+00 7.47589571e-01]]

 =  [[-0.80618188 -0.54157827 -0.23825146]
 [-0.45661854  0.31342349  0.8326255 ]
 [ 0.37625828 -0.78003762  0.49997102]]


COM O TERMO INDEPENDENTE: 
V =  [[-0.79610972  0.1628056  -0.47792329 -0.33360603]
 [-0.53478061 -0.03864172  0.33762985  0.77364243]
 [-0.23529777  0.18572124  0.80860497 -0.50626137]
 [-0.15765144 -0.96825037  0.06126507 -0.18407563]]

!$\Lambda$! =  [[5.86392634e+02 0.00000000e+00 0.00000000e+00]
 [0.00000000e+00 5.29776824e-01 0.00000000e+00]
 [0.00000000e+00 0.00000000e+00 7.47589571e-01]]

V^T =  [[-0.79610972 -0.53478061 -0.23529777 -0.15765144]
 [ 0.1628056  -0.03864172  0.18572124 -0.96825037]
 [-0.47792329  0.33762985  0.80860497  0.06126507]
 [-0.33360603  0.77364243 -0.50626137 -0.18407563]]

\end{lstlisting}

\subsubsection{Iris-Versicolor}

\begin{lstlisting}
Iris-Versicolor

SEM O TERMO INDEPENDENTE: 
V =  [[ 0.76039163  0.64759888 -0.0491961 ]
 [ 0.35223975 -0.47485674 -0.80649751]
 [ 0.54564799 -0.59592513  0.58918716]]

!$\Lambda$! =  [[9.59570396e+02 0.00000000e+00 0.00000000e+00]
 [0.00000000e+00 1.33162776e+00 0.00000000e+00]
 [0.00000000e+00 0.00000000e+00 9.57976568e-01]]

V^T =  [[ 0.76039163  0.35223975  0.54564799]
 [ 0.64759888 -0.47485674 -0.59592513]
 [-0.0491961  -0.80649751  0.58918716]]


COM O TERMO INDEPENDENTE: 
V =  [[-0.7545525   0.11812875  0.64474223  0.03167953]
 [-0.34954066  0.11526607 -0.46962259  0.80248968]
 [-0.54144512 -0.01367345 -0.60234554 -0.58637025]
 [-0.1237297  -0.98619084  0.030691    0.1057197 ]]

!$\Lambda$! =  [[9.59570396e+02 0.00000000e+00 0.00000000e+00]
 [0.00000000e+00 1.33162776e+00 0.00000000e+00]
 [0.00000000e+00 0.00000000e+00 9.57976568e-01]]

V^T =  [[-0.7545525  -0.34954066 -0.54144512 -0.1237297 ]
 [ 0.11812875  0.11526607 -0.01367345 -0.98619084]
 [ 0.64474223 -0.46962259 -0.60234554  0.030691  ]
 [ 0.03167953  0.80248968 -0.58637025  0.1057197 ]]
\end{lstlisting}

\subsubsection{Iris-Virginica}

\begin{lstlisting}
Iris-Virginica

SEM O TERMO INDEPENDENTE: 
V =  [[-0.72592866 -0.63483621  0.2645951 ]
 [-0.32772296 -0.01894783 -0.94458385]
 [-0.60466953  0.77241438  0.19429563]]

!$\Lambda$! =  [[1.28249881e+03 0.00000000e+00 0.00000000e+00]
 [0.00000000e+00 5.79253032e-01 0.00000000e+00]
 [0.00000000e+00 0.00000000e+00 1.07193335e+00]]

V^T =  [[-0.72592866 -0.32772296 -0.60466953]
 [-0.63483621 -0.01894783  0.77241438]
 [ 0.2645951  -0.94458385  0.19429563]]


COM O TERMO INDEPENDENTE: 
V =  [[-7.21743001e-01 -2.76297507e-01  6.34623278e-01 -1.50933108e-04]
 [-3.25843186e-01  9.35769444e-01  3.68031853e-02 -1.29642937e-01]
 [-6.01187515e-01 -1.93722166e-01 -7.68083597e-01 -1.05322749e-01]
 [-1.07176625e-01  1.02308151e-01 -7.71129602e-02  9.85951218e-01]]

!$\Lambda$! =  [[1.28249881e+03 0.00000000e+00 0.00000000e+00]
 [0.00000000e+00 5.79253032e-01 0.00000000e+00]
 [0.00000000e+00 0.00000000e+00 1.07193335e+00]]

V^T =  [[-7.21743001e-01 -3.25843186e-01 -6.01187515e-01 -1.07176625e-01]
 [-2.76297507e-01  9.35769444e-01 -1.93722166e-01  1.02308151e-01]
 [ 6.34623278e-01  3.68031853e-02 -7.68083597e-01 -7.71129602e-02]
 [-1.50933108e-04 -1.29642937e-01 -1.05322749e-01  9.85951218e-01]]
\end{lstlisting}

\section{Questão 3}
\subsection{Resultados}
\subsubsection{Iris-Setosa}
\begin{lstlisting}
Iris-Setosa

SEM O TERMO INDEPENDENTE: 
U =  [[-0.79610972  0.33360603  0.47792329 -0.1628056 ]
 [-0.53478061 -0.77364243 -0.33762985  0.03864172]
 [-0.23529777  0.50626137 -0.80860497 -0.18572124]
 [-0.15765144  0.18407563 -0.06126507  0.96825037]]

!$\Sigma$! =  [6.01336860e+02 7.73257486e-01 5.31947766e-01 2.79352229e-02]

V^T =  [[-0.79610972 -0.53478061 -0.23529777 -0.15765144]
 [ 0.33360603 -0.77364243  0.50626137  0.18407563]
 [ 0.47792329 -0.33762985 -0.80860497 -0.06126507]
 [-0.1628056   0.03864172 -0.18572124  0.96825037]]


COM O TERMO INDEPENDENTE: 
U =  [[-0.79610972  0.33360603  0.47792329 -0.1628056 ]
 [-0.53478061 -0.77364243 -0.33762985  0.03864172]
 [-0.23529777  0.50626137 -0.80860497 -0.18572124]
 [-0.15765144  0.18407563 -0.06126507  0.96825037]]

!$\Sigma$! =  [6.01336860e+02 7.73257486e-01 5.31947766e-01 2.79352229e-02]

V^T =  [[-0.79610972 -0.53478061 -0.23529777 -0.15765144]
 [ 0.33360603 -0.77364243  0.50626137  0.18407563]
 [ 0.47792329 -0.33762985 -0.80860497 -0.06126507]
 [-0.1628056   0.03864172 -0.18572124  0.96825037]]
\end{lstlisting}

\subsubsection{Iris-Versicolor}
\begin{lstlisting}
Iris-Versicolor

SEM O TERMO INDEPENDENTE: 
U =  [[-0.7545525   0.64474223  0.03167953 -0.11812875]
 [-0.34954066 -0.46962259  0.80248968 -0.11526607]
 [-0.54144512 -0.60234554 -0.58637025  0.01367345]
 [-0.1237297   0.030691    0.1057197   0.98619084]]

!$\Sigma$! =  [9.74487597e+02 1.33280153e+00 9.68188359e-01 7.14132919e-02]

V^T =  [[-0.7545525  -0.34954066 -0.54144512 -0.1237297 ]
 [ 0.64474223 -0.46962259 -0.60234554  0.030691  ]
 [ 0.03167953  0.80248968 -0.58637025  0.1057197 ]
 [-0.11812875 -0.11526607  0.01367345  0.98619084]]


COM O TERMO INDEPENDENTE: 
U =  [[-0.7545525   0.64474223  0.03167953 -0.11812875]
 [-0.34954066 -0.46962259  0.80248968 -0.11526607]
 [-0.54144512 -0.60234554 -0.58637025  0.01367345]
 [-0.1237297   0.030691    0.1057197   0.98619084]]

!$\Sigma$! =  [9.74487597e+02 1.33280153e+00 9.68188359e-01 7.14132919e-02]

V^T =  [[-0.7545525  -0.34954066 -0.54144512 -0.1237297 ]
 [ 0.64474223 -0.46962259 -0.60234554  0.030691  ]
 [ 0.03167953  0.80248968 -0.58637025  0.1057197 ]
 [-0.11812875 -0.11526607  0.01367345  0.98619084]]
\end{lstlisting}

\subsubsection{Iris-Virginica}
\begin{lstlisting}
Iris-Virginica

SEM O TERMO INDEPENDENTE: 
U =  [[-7.21743001e-01  2.76297507e-01  6.34623278e-01 -1.50933108e-04]
 [-3.25843186e-01 -9.35769444e-01  3.68031853e-02 -1.29642937e-01]
 [-6.01187515e-01  1.93722166e-01 -7.68083597e-01 -1.05322749e-01]
 [-1.07176625e-01 -1.02308151e-01 -7.71129602e-02  9.85951218e-01]]

!$\Sigma$! =  [1.29740071e+03 1.08243546e+00 5.82311582e-01 8.45463191e-02]

V^T =  [[-7.21743001e-01 -3.25843186e-01 -6.01187515e-01 -1.07176625e-01]
 [ 2.76297507e-01 -9.35769444e-01  1.93722166e-01 -1.02308151e-01]
 [ 6.34623278e-01  3.68031853e-02 -7.68083597e-01 -7.71129602e-02]
 [-1.50933108e-04 -1.29642937e-01 -1.05322749e-01  9.85951218e-01]]


COM O TERMO INDEPENDENTE: 
U =  [[-7.21743001e-01  2.76297507e-01  6.34623278e-01 -1.50933108e-04]
 [-3.25843186e-01 -9.35769444e-01  3.68031853e-02 -1.29642937e-01]
 [-6.01187515e-01  1.93722166e-01 -7.68083597e-01 -1.05322749e-01]
 [-1.07176625e-01 -1.02308151e-01 -7.71129602e-02  9.85951218e-01]]

!$\Sigma$! =  [1.29740071e+03 1.08243546e+00 5.82311582e-01 8.45463191e-02]

V^T =  [[-7.21743001e-01 -3.25843186e-01 -6.01187515e-01 -1.07176625e-01]
 [ 2.76297507e-01 -9.35769444e-01  1.93722166e-01 -1.02308151e-01]
 [ 6.34623278e-01  3.68031853e-02 -7.68083597e-01 -7.71129602e-02]
 [-1.50933108e-04 -1.29642937e-01 -1.05322749e-01  9.85951218e-01]]
\end{lstlisting}

\section{Questão 4}
\subsection{Resultados}

\section{Código}




\section{Bibliografia}
\begin{itemize}
   \item https://algebralinearufcg.github.io/jup-not/prog02-learning-numpy.html
   \item https://machinelearningmastery.com/singular-value-decomposition-for-machine-learning/
   \item https://pt.coredump.biz/questions/34007632/how-to-remove-a-column-in-a-numpy-array
   \item https://pythonforundergradengineers.com/unicode-characters-in-python.html
 \end{itemize}




\begin{lstlisting}


			
\end{lstlisting}

\end{document}
