\documentclass[a4paper,12pt,twoside]{article}
\usepackage{xcolor}
\definecolor{verde}{rgb}{0,0.5,0}
\definecolor{jpurple}{rgb}{0.5,0,0.35}
\usepackage[utf8]{inputenc}
\usepackage{graphicx}
\usepackage{float}
\usepackage[brazilian]{babel}
\usepackage{indentfirst}
\usepackage{steinmetz}
\usepackage{multicol}
\usepackage[left=2cm, right=2cm, top=2cm]{geometry}
\setlength\parindent{1cm}
\usepackage{mathrsfs, amsmath}
\usepackage{textcomp}
\usepackage{gensymb}
\usepackage{lipsum}
\usepackage{natbib}
\usepackage{listings}
\usepackage{xcolor}



% Definindo novas cores
\definecolor{verde}{rgb}{0.25,0.5,0.35}
\definecolor{jpurple}{rgb}{0.5,0,0.35}
% Configurando layout para mostrar codigos Python

\lstset{
  language=Python,
  basicstyle=\ttfamily\small,
  keywordstyle=\color{jpurple}\bfseries,
  stringstyle=\color{red},
  commentstyle=\color{verde},
  morecomment=[s][\color{blue}]{/**}{*/},
  extendedchars=true,
  showspaces=false,
  showstringspaces=false,
  numbers=left,
  numberstyle=\tiny,
  breaklines=true,
  backgroundcolor=\color{cyan!10},
  breakautoindent=true,
  captionpos=b,
  xleftmargin=0pt,
  tabsize=4
}



\headheight = 10pt

\setcounter{section}{-1}



\date{}

\begin{document}

% capa
\begin{titlepage} %iniciando a "capa"
\begin{center} %centralizar o texto abaixo
{\large Universidade Federal do Rio de Janeiro}\\[0.2cm] %0,2cm é a distância entre o texto dessa linha e o texto da próxima
{\large Escola Politécnica}\\[0.2cm] % o comando \\ "manda" o texto ir para próxima linha
{\large Departamento de Engenharia Eletrônica e de Computação}\\[0.2cm]
{\large Engenharia Eletrônica e de Computação}\\[0.2cm]
{\large Álgebra Linear 2}\\[5.1cm]
{\bf \huge TRABALHO DE}\\ % o comando \bf deixa o texto entre chaves em negrito. O comando \huge deixa o texto enorme
{\bf \huge ÁLGEBRA LINEAR 2}\\[5.1cm] 
\end{center} %término do comando centralizar
{\large Aluna: Karen dos Anjos Arcoverde}\\[0.7cm] % o comando \large deixa o texto grande
{\large Professor: Marcello Luiz Rodrigues de Campos}\\[5.1cm]
\begin{center}
{\large Rio de Janeiro}\\[0.2cm]
{\large 2021}
\end{center}
\end{titlepage} %término da "capa"


\renewcommand{\contentsname}{Sumário}

\tableofcontents
\clearpage





\section{Introdução}
\subsection{Conteúdo}
    O relatório contém os resultados encontrados para cada questão passada pelo professor e o código final em linguagem Python.
\subsection{Software e linguagem}
    O software usado para programação foi o Spyder e a linguagem foi o Python 3.8.
\subsection{Bibliotecas}
As bibiotecas utilizadas para construir o código em Python foram:
\begin{itemize}
   \item Numpy
   \item Scipy
 \end{itemize}
 \subsection{Base de dados}
 O  conjunto de dados "Iris" selecionado para o trabalho foi:
 \begin{table}[H]
\begin{tabular}{|c|c|c|} \hline
\textbf{ESPÉCIES} & DE & PARA \\\hline
\textbf{Iris-setosa}  & 25 & 39 \\\hline
\textbf{Iris-versicolor} & 75 & 89  \\\hline
\textbf{Iris-virginica} & 125 & 139 \\\hline

 
\end{tabular}
\label{tabela2}
\centering
\caption{Base de dados "Iris" selecionada}
\label {tabela2}
\end{table}



\section{Questão 1}
\subsection{Resultado}

\section{Questão 2}
\subsection{Resultado}

\section{Questão 3}
\subsection{Resultado}


\section{Questão 4}
\subsection{Resultado}

\section{Código}




\section{Bibliografia}
\begin{itemize}
   \item https://algebralinearufcg.github.io/jup-not/prog02-learning-numpy.html
   \item https://machinelearningmastery.com/singular-value-decomposition-for-machine-learning/
 \end{itemize}




\begin{lstlisting}


			
\end{lstlisting}

\end{document}
